\documentclass[platex]{suribt}
\usepackage{url}
\usepackage{graphicx}
\usepackage[dvipdfmx]{color}
\title{ランダムフォレストによるサッカーW杯の優勝国予想}
\author{石田亜斗武}
\eauthor{Atomu Ishida}% Copyright 表示で使われる
\studentid{2014011}
\supervisor{六井 淳 教授\deputy 渡邊貴之 教授}% 1 つ引数をとる (役職まで含めて書く)
\handin{2024}{1}% 提出月. 2 つ (年, 月) 引数をとる

\begin{document}
\maketitle%%%%%%%%%%%%%%%%%%% タイトル %%%%


\frontmatter% ここから前文
\begin{jabstract}%%%%%%%%%%%%% 概要 %%%%%%%%
  スポーツの勝敗予測は、選手のコンディション、戦術の変化、チームの相互作用など、多数の変数に影響される。本研究では、私自身の長年のサッカーの経験と、人工知能や機械学習の知識を使い、FIFAワールドカップ優勝国の予測を行う。\par
  本研究ではスポーツの勝敗予測における変数の複雑性に対応するため、ランダムフォレストという機械学習手法を用いる。ランダムフォレストはモデル構築とチューニングの容易さから、活用範囲の広い手法であることが知られている。FIFAワールドカップの優勝国を予測するために過去の大会の結果からデータセットを作成し、ランダムフォレストで予測モデルを構築し予測を行った。

\end{jabstract}

\newpage


\begin{eabstract}%%%%%%%%%%%%% 概要 %%%%%%%%
  Predicting winners and losers in sports is extremely difficult because it is influenced by numerous variables, such as player conditions, tactical changes, and team interactions. In recent years, statistical methods and machine learning have been used to improve the accuracy of predicting the outcome of difficult-to-predict sports matches. Therefore, in this study, I will use my own many years of soccer experience and knowledge of artificial intelligence and machine learning to predict the FIFA World Cup winning country.\par
  To cope with the complexity of variables in sports win/loss prediction, this study uses a machine learning method called random forests. Compared to other methods, Random Forest captures the diversity and complexity of the data by utilizing a large number of decision trees, reducing the risk of overlearning and providing reliable prediction results. In addition, it is easy to build and tune models, enabling results to be achieved within a limited research timeframe. The results of this research are expected to provide useful information for those involved in the FIFA World Cup in response to the growing use of data in the field of sports analysis. 
\end{eabstract}

\setcounter{tocdepth}{2}
\tableofcontents%%%%%%%%%%%%% 目次 %%%%%%%%
\mainmatter% ここから本文 %%% 本文 %%%%%%%%
\chapter{はじめに}
\section{研究背景}
スポーツの勝敗予測は、選手のコンディション、戦術の変化、チームの相互作用など、多数の変数に影響されるため、予測が非常に困難である[1]。近年では、予測が困難なスポーツの試合結果を科学的に分析するアナリティクスの分野が注目され、予測の精度を高めるために統計的手法や機械学習が利用されるようになっている[2]。本研究では、私自身の長年のサッカーの経験と、人工知能や機械学習の知識を使い、FIFAワールドカップ優勝国の予測を行う。
\section{研究目的}
本研究の目的は、過去のFIFAワールドカップのデータをもとに高精度な予測モデルを構築し、優勝国を予測することである。サッカーの試合結果予測においては、様々な機械学習手法が採用されている。イングランドのプロサッカーリーグの試合結果予測を行った研究[3]では、決定木[4]によってデータの特徴間の関係をモデル化しやすくし、ナイーブベイズ[5]によって特徴間の独立性を仮定することで、サッカーのような多くの変数が影響するデータに対応し予測を行っている。また同じくイングランドのプロサッカーリーグに対して試合結果予測を行った別の研究[6]では、サポートベクタマシン[7]の過学習[8]を防ぎながら限られたデータからでも高い予測精度を達成するという特性を活かして予測を行っている。\par

本研究ではこれらの手法の中からランダムフォレスト[9]を採用する。ランダムフォレストは、複数の決定木を組み合わせることでデータの多様性と複雑な関連性を捉え、過学習のリスクを減らすことができる[10][11]ため、ナイーブベイズの変数同士の複雑な関係やデータパターンを捉えるのには限界があるという欠点[12]を解決することでき、決定木の訓練データに過剰に適合する傾向がある[13]という問題を解決することができる。また、ランダムフォレストにはモデル構築とチューニングの容易であるという特徴があり[14]、サポートベクタマシンのモデル設計やチューニングが困難であるという欠点[15]を解決することができる。\par

さらに、本研究ではランダムフォレストを用いた先行研究[16][17]では取り扱わないジニ係数[18]を特徴量としてを加える。ジニ係数とは国の貧富格差を示す指標であり、貧困がサッカーにおけるスキルの習得と知覚運動スキルの発達にプラスの影響を与えるといった研究[19]もある。これは、従来のサッカーの勝敗予測ではあまり注目されてこなかった選手の心理的側面を考慮することにより、試合結果に影響を与える可能性のある新たな要因をモデルに組み込む試みである。このアプローチにより、単に試合結果の統計的な情報だけでなく、選手の心理状態も試合結果予測の重要な要素として取り入れることができると考える。ランダムフォレストを用いたクラス分類によってFIFAワールドカップの優勝国予測を目的とした検証実験を行う。
\section{論文の構成}
第2章では、本研究の関連技術と用語について述べる。\par
第3章では、検証実験について述べる。\par
第4章では、まとめと今後について述べる。

\chapter{関連技術}
本章では、本研究に関連する技術について述べる。
\section{ランダムフォレスト}
ランダムフォレストは、決定木を基にした集団学習アルゴリズムである。この手法では、多数の決定木がデータの異なるランダムなサブセットに基づいて訓練され、それぞれの出力に基づいて最終的な予測結果が多数決または平均値によって決定される(図2.1)。ランダムフォレストの大きな特徴は、個々の決定木の学習にランダム性を導入している点にある。これによって、各決定木が相補的に機能し、過学習を防ぎつつ高い汎化性能を得ることが可能となる。

\includegraphics[scale=0.6]{randomforest.png}

ランダムフォレストの大きな特徴は、個々の決定木の学習にランダム性を導入している点にある。これによって、各決定木が相補的に機能し、過学習を防ぎつつ高い汎化性能を得ることが可能となる。いかにその流れを説明する。

ランダムフォレストの学習メカニズムは、ランダム性を2つの主要な側面で取り入れる。まず、バギング(Bootstrap Aggregating)を用いて、各決定木の訓練データを選定する。これは、元のデータセット \(S\) から重複を許してランダムにサンプル \(S_{0t}\) を抽出するプロセスである。次に、決定木の各ノードでの分割関数の学習にランダム性を導入する。分割関数のパラメータ \(\theta_j\) に対する候補集合 \(\tau\) は、一般に広範囲にわたるが、そのサブセット \(\tau_j \subseteq \tau\) のみを使用し、その選択をランダムに行う。

ランダム性の度合いは、比率 \(|\tau_j| / |\tau|\) によって調整可能である。ここで、\(\rho = |\tau_j|\) はランダム性の指標として機能し、\(\rho\) の値に応じて、木間の相関が変化する。例えば、\(\rho = |\tau|\) の場合、すべての決定木で同一の学習データが使用されるため、ランダム性はない。逆に、\(\rho = 1\) の場合、完全にランダムな選択が行われ、互いに相関のない決定木が得られる。

各決定木の構築では、根ノードから始めて、最適な分割関数を選択し、データを左右の子ノードに割り当てる。この過程は、すべてのノードにおいて再帰的に繰り返される。具体的には、ノード番号 \(i\) のデータセット \(S_i\) が分割関数によって分割され、左右の子ノードに対応するデータセット \(S_{iL}\) と \(S_{iR}\) に割り当てられる。このプロセスは、決定木のすべてのノードで順に実行され、各ノードで \(S_j = S_{jL} \cup S_{jR}\) および \(S_{jL} \cap S_{jR} = \emptyset\)、\(S_{jL} = S_{2j+1}\)、\(S_{jR} = S_{2j+2}\) が成立する。

分割関数 \(h(v, \theta_j)\) の学習は、ノード \(j\) における最適なパラメータ \(\theta_j^*\) を選択するプロセスであり、以下のように定義される:
\begin{equation}
\theta_j^* = \arg \max_{\theta_j \in \tau_j} I_j(S_j, S_{jL}, S_{jR}, \theta_j)
\end{equation}
ここで、\(I_j\) は情報利得を表す目的関数である。情報利得は以下のように計算される:
\begin{equation}
I_j = H(S_j) - \sum_{i \in \{L, R\}} \frac{|S_{ji}|}{|S_j|} H(S_{ji})
\end{equation}
ここで、\(H(S)\) はデータセット \(S\) のエントロピーであり、\(S_{ji}\) はノード \(j\) の左右の子ノードに割り当てられるデータセットを示す。

以上の学習アルゴリズムは、決定木のノードに割り当てられる学習データ集合の数が1になるまで継続される。ただし、過学習を防ぐために適切な停止条件を設定する必要がある。そのための方法としては、あらかじめ設定した最大高さ \(D\) に達した場合、ノードに割り当てられた学習データの個数が一定値以下になった場合、または分割による情報利得が一定値以下になった場合などがある。

\subsection{特徴選択}
ランダムフォレストにおける特徴量選択は、モデルの予測性能と解釈性を向上させるための重要なプロセスである。各決定木がランダムに選択された特徴量のサブセットを使用することで、データセットの異なる側面を捉え、全体としてバランスの取れた分析を行うことができる。

特徴量の重要度の計算においては、ジニ不純度がキーとなる指標である。ジニ不純度 \( G(k) \) は、特定のノードにおけるサンプルの分布の均一性を測定する。式
\begin{equation}
G(k) = \sum_{i=1}^{n} p(i) \times (1-p(i))
\end{equation}
は、ノード \( k \) におけるターゲットラベル \( i \) の出現確率 \( p(i) \) を用いて、そのノードの不純度を算出する。完全に均一なサンプルの集合(すべてが同じラベル)の場合、ジニ不純度は0となり、サンプルが多様なラベルを持つほどジニ不純度は高くなる。

特徴量の重要度 \( I(j) \) は、特徴量がデータの分割にどれだけ効果的に寄与しているかを示す。特徴量 \( j \) を用いてノードを分割することで得られるジニ不純度の減少量を、すべてのノードにわたって集計する。この計算は、式
\begin{equation}
I(j) = \sum_{i=1}^{n \in F(j)} (N_{parent}(i) \times G_{parent}(i)) - (N_{left\*child}(i) \times G*{left\*child}(i) + N*{right\*child}(i) \times G*{right\*child}(i))
\end{equation}
によって表される。ここで、\( F(j) \) は特徴量 \( j \) が分割に使用されるノードの集合を示し、\( N*{parent}(i) \)、\( N_{left\*child}(i) \)、\( N*{right\*child}(i) \) はそれぞれ親ノードとその子ノードのサンプル数を表す。\( G*{parent}(i) \)、\( G_{left\*child}(i) \)、\( G*{right\_child}(i) \) はそれぞれのノードにおけるジニ不純度を示す。

このようにして計算された特徴量の重要度は、モデルにおける各特徴量の寄与度を示し、モデルの解釈に役立つ。特に、予測結果に大きな影響を与える特徴量を特定することで、より効果的な特徴量エンジニアリングやモデルの改善に繋がる。また、データセット内の特徴量間の関係性を理解する上で重要な手がかりを提供する。
\section{データの前処理}
\section{ハイパーパラメータチューニング}

\chapter{検証実験}
\section{データ詳細}
\section{評価指標}
\subsection{混合行列}
\subsection{正解率}
\subsection{適合率}
\subsection{再現率}
\subsection{F値}
\section{検証内容}
\section{検証結果}
\section{考察}

\chapter{まとめと今後の課題}
\section{まとめ}
\section{今後の課題}

\backmatter% ここから後付
\chapter{謝辞}%%%%%%%%%%%%%%% 謝辞 %%%%%%%

\begin{thebibliography}{2}%%%% 参考文献 %%%
\bibitem{}
Milad Keshtkar Langaroudi, Mohammad Reza Yamaghani, “Sports Result Prediction Based on Machine Learning and Computational Intelligence Approaches A Survey”, P1, 2019
\bibitem{}
谷岡広樹, "スポーツアナリティクスにおけるデータとAI活用",P1, 2020
\bibitem{}
Saurabh Vaidya ,”Football Match Winner Prediction”, P1,P2, 2016
\bibitem{}
Lucidspark, "決定木分析とは?メリットとやり方", \url{https://lucidspark.com/ja/blog/how-to-make-a-decision-tree}
\bibitem{}
AVINTON ,”機械学習入門者向け Naive Bayes(単純ベイズ)アルゴリズムに触れてみる”,\url{https://avinton.com/academy/naive-bayes/}
\bibitem{}
Chinwe Peace Igiri,"Support Vector Machine-Based Prediction System for a Football Match Result", P1, 2015
\bibitem{}
AIsmiley, "サポートベクターマシン(SVM)とは?特徴やメリットと活用事例", \url{https://aismiley.co.jp/ai_news/svm/}
\bibitem{}
TRYETING"機械学習における過学習とは何か?原因・回避方法をくわしく解説", \url{https://www.tryeting.jp/column/6846/}
\bibitem{}
IBM"ランダムフォレストとは",\url{https://www.ibm.com/jp-ja/topics/random-forest}
\bibitem{}
Matthias Schonlau, Rosie Yuyan Zou, "The random forest algorithm for statistical learning", P4, 2020
\bibitem{}
Kai Liang,"Analysis and Evaluation of Sports Effect Based on Random Forest Algo-rithm under Big Data",P2, 2022
\bibitem{}
OpenGenus"9Advantages and 10disadvantages of Naive Bayes Algorithm", \url{https://iq.opengenus.org/advantages-and-disadvantages-of-naive-bayes-algorithm/}
\bibitem{}
InsideLearningMachines"8 Key Advantages And Disadvantages Of Decision Trees”, \url{https://insidelearningmachines.com/advantages_and_disadvantages_of_decision_trees/}
\bibitem{}
”Hyperparameters and Tuning Strategies for Random Forest”, \url{https://ar5iv.labs.arxiv.org/html/1804.03515}, 2019
\bibitem{}
Rosita Guido, ”A hyper-parameter tuning approach for cost-sensitive support vector machine classifiers”,P3,2022
\bibitem{}
Ayush Majumdar,\url{https://ieeexplore.ieee.org/author/37089837731},”Football Match Prediction using Exploratory Data Analysis and Multi-Output Regression”,P2, 2022
\bibitem{}
Pakawan Pugsee, ”Football Match Result Prediction Using the Random Forest Classifier”,P2, 2019
\bibitem{}
野村證券”ジニ係数|証券用語解説集”, \url{https://www.nomura.co.jp/terms/japan/si/A02571.html}
\bibitem{}
Luiz Ueharaほか.” The Poor “Wealth” of Brazilian Football: How Poverty May Shape Skill and Expertise of Players”. \url{https://www.frontiersin.org/articles/10.3389/fspor.2021.635241/}


\end{thebibliography}
\appendix% ここから付録 %%%%% 付録 %%%%%%%
\chapter{}
\end{document}